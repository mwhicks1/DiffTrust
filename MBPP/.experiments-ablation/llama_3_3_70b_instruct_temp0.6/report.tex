\documentclass[12pt]{article}%
\usepackage[T1]{fontenc}%
\usepackage[utf8]{inputenc}%
\usepackage{lmodern}%
\usepackage{textcomp}%
\usepackage{lastpage}%
\usepackage{graphicx}%
%
%
%
\begin{document}%
\normalsize%
\section{Incoherence{-}Based Experiment Analysis}%
\label{sec:Incoherence{-}BasedExperimentAnalysis}%
This report presents a statistical analysis of the model’s performance across tasks, focusing on the relationship between incoherence scores (Incoherence) and execution errors (Error).%
\newline%
\newline%
Number of tasks analyzed: 404%
\section{Introduction}%
\label{sec:Introduction}%
This report summarizes the results of an automatic evaluation of code generation using the following configuration parameters.%
\newline%
\begin{tabular}{ll}%
\textbf{Parameter}&\textbf{Value}\\%
\hline%
Language Model&llama\_3\_3\_70b\_instruct\\%
Temperature&0.6\\%
\$m\$ (number of candidates)&10\\%
\$n\$ (number of samples used to estimate metrics)&1000\\%
Timeout per metric estimation (s)&60.0\\%
\end{tabular}%
\newline%
The model was tested across a suite of programming tasks. We aim to explore how the model’s incoherence signal relates to execution{-}time failures.

%
\subsection{Summary Statistics}%
\label{subsec:SummaryStatistics}%
\begin{tabular}{lcccc}%
Metric&Mean&Std&Min&Max\\%
Raw Incoherence&0.166&0.217&0.000&0.869\\%
Raw Error&0.351&0.373&0.000&1.000\\%
\end{tabular}

%
\subsection{Error Detection Analysis}%
\label{subsec:ErrorDetectionAnalysis}%
\begin{tabular}{lc}%
Metric&Value\\%
Errors (Error > 0)&309\\%
Error Rate&76.49\%\\%
Detected Errors (Error > 0 and Incoherence > 0)&254\\%
Detection Rate&82.20\%\\%
Confident (Incoherence = 0)&148\\%
Confident Error Count&55\\%
Confident Error Rate&37.16\%\\%
Mean Error When Confident&0.2138\\%
\end{tabular}

%
\subsection{Correlation Analysis}%
\label{subsec:CorrelationAnalysis}%
\begin{tabular}{lcccc}%
Metric&Pearson r&Pearson p&Spearman \$\textbackslash{}rho\$&Spearman p\\%
Incoherence vs Error&0.486&2.669e{-}25&0.558&2.081e{-}34\\%
\end{tabular}

%
\newpage%
\subsection{Bubble Plot of Incoherence and Error}%
\label{subsec:BubblePlotofIncoherenceandError}%
This plot shows the density of (Incoherence, Error) points using bubble size to indicate frequency.%


\begin{figure}[htbp]%
\centering%
\includegraphics[width=0.75\linewidth]{C:/Users/Thomas Valentin/Desktop/StageM1/Python/v1/MBPP/.experiments-ablation/llama_3_3_70b_instruct_temp0.6/plot/bubble.png}%
\caption{Bubble Plot: Incoherence vs Error}%
\end{figure}

%
\subsection{Log{-}Log Plot of Incoherence and Error}%
\label{subsec:Log{-}LogPlotofIncoherenceandError}%
This plot displays the relationship between Incoherence and Error in log{-}log scale. Only data points where both values are strictly positive are included.%


\begin{figure}[htbp]%
\centering%
\includegraphics[width=0.75\linewidth]{C:/Users/Thomas Valentin/Desktop/StageM1/Python/v1/MBPP/.experiments-ablation/llama_3_3_70b_instruct_temp0.6/plot/log_log.png}%
\caption{Log{-}Log Scatter Plot: Incoherence vs Error}%
\end{figure}

%
\end{document}